\section{Introduction}\label{introduction}

Seismic design and assessment of tall and flexible structures or
seismically isolated systems usually require spectra computed with a
critical damping percentage different from the 5\% contained in most
building codes\citeproc{ref-kawasumi1956}{{[}1{]}}. Even though the
exact dynamical response of a viscously damped elastic structure is
obtained by solving the equations of motion for a given record, rarely
does the engineer have access to the records that generated the design
spectrum. Therefore, to design structures with critical damping
different than 5\%, the engineer must either estimate the spectral
ordinates directly using analytical methods, e.g.~with random vibration
theory, or use an empirical multiplicative factor. This factor is
referenced as a `damping scaling factor' or `damping modification
factor' (DMF or B in the literature) and is defined as the spectral
ordinate ratio for an arbitrary damping with respect to the 5\% damped
ordinates and will be referenced henceforth as DMF.

Most authors acknowledge DMF dependence on various physical parameters
of both the system and ground motion, including damping, system
frequency, magnitude, fault distance, energy dissipation cycles,
frequency content, and soil conditions. However, most studies neglect
the influence of damping effects in DMF's peaks versus valleys, and
statistical regression models average this effect, particularly for high
frequencies.

Hudson\citeproc{ref-hudson1956}{{[}2{]}} conducted one of the earliest
analytical studies on response spectrum techniques in earthquake
engineering, deriving a mean velocity spectrum for randomly distributed
vibrations (see their Appendix for details).

Empirical DMF computation studies trace back to the pioneering work of
Newmark and Hall\citeproc{ref-newmark1982}{{[}3{]}}, who used a small
number of ground motions to find clear dependency on vibration period
and proposed equations for all spectral regions.

Arias and Husid\citeproc{ref-arias1962}{{[}4{]}} used random vibration
theory to show that under certain assumptions, the DMF is proportional
to the square root of the damping ratio. However, finite duration and
stochasticity of real earthquakes significantly influence system
response. This led them to suggest an exponent of 0.4 instead of 0.5,
representing one of the earliest analytical treatments of this predictor
variable. Rosenblueth\citeproc{ref-rosenblueth1964}{{[}5{]}} validated
this result and demonstrated strong dependence on motion duration,
citing the 1960 Agadir earthquake as an example of a destructive,
short-duration earthquake.

Using a different approach, Miranda and
Miranda\citeproc{ref-miranda2020}{{[}6{]}} found that a measure of
spectral shape called \emph{SaRatio} (\(S_R\) henceforth) is a better
statistical predictor for the DMF than magnitude, distance,
record-duration, and other physical event-specific parameters. This
makes intuitive sense since it is natural to think that all
peculiarities of the motion due to event-specific parameters should
manifest themselves in the shape of the spectrum. This is a notable
example of a study based on analytical methods that does not use random
vibration theory. A study on the effectiveness of this measure for
near-fault pulse-like ground motions, however, has not been performed to
date.

The destructive effects of the 1994 Northridge and 1995 Kobe earthquakes
renewed interest in seismologists and engineers in developing parametric
models that can capture the qualitative nature of the displacement,
velocity, and acceleration pulses that are characteristic of the rupture
processes of motions with forward directivity. These motions contain a
dominant velocity pulse that delivers an energy burst to structures,
which causes significant plastic displacements. Such motions are
referred to as `near-fault' and `pulse-like' motions, or shortly as
NFPL.

Hubbard and Mavroeidis\citeproc{ref-hubbard2011}{{[}7{]}} studied the
influence of earthquake magnitude on the DMF for NFPLs and observed that
its peak clearly depended on the magnitude, which might be playing as a
proxy to duration. Damping modification factors for high-magnitude
earthquakes show a smaller peak at 0.75s which is associated with
high-frequency radiation that is very difficult to fully capture in a
model. Most importantly, they showed that it strongly depends on the
pulse period and that a sounder approach is to regress towards \(T/T_p\)
rather than the oscillator period alone. They proposed a more accurate
estimator to account for this phenomenon:

\[\frac{1}{DMF} = 3.4\frac{\xi^{1.3}}{(T/T_p)^{1.3}} + 1\qquad 0.10 \leq \xi \leq 0.50\]

and

\[\frac{1}{DMF} = 2\frac{(\xi + 0.3)^{1.5}}{(T/T_p)^{1.3}} + 1\qquad 0.50 < \xi \leq 1.00\]

which is representative down until \(T/T_p=0.83\), a normalized period
where the peak in value seems to occur. Below this normalized period,
the factor is linearly reduced to 1 as required by structural dynamics.

Pu et al.\citeproc{ref-pu2016}{{[}8{]}} pointed out the possible
inaccuracy of many building codes recommending generic estimators when
the motions the engineer is designing for are near fault with a
potentially large velocity waveform. Their findings agree with past
research in that the DMF presents significant dependence on the pulse
period \(T_p\) and that there is some dependence on magnitude and soil
conditions. Notably, they investigated the effect of the impulsive
character of near-fault motions on the velocity and acceleration spectra
(and not only on the displacement spectra). On this matter, they
concluded that the spectra are different enough that they must be
computed individually and cannot be derived one from another.
Furthermore, they present a statistical predictor based on 50
hand-picked pulse-like near-fault ground motions. These pulses were
selected using an analytical approach that verifies the presence of a
dominant velocity pulse. The displacement estimator presented is:

\[\frac{1}{DMF} = 3.4\frac{\xi^{1.3}}{(T/T_p)^{1.3}} + 1\qquad 0.10 \leq \xi \leq 0.50\]

and

\[\frac{1}{DMF} = 2\frac{(\xi + 0.3)^{1.5}}{(T/T_p)^{1.3}} + 1\qquad 0.50 < \xi \leq 1.00\]

Mollaioli et al.\citeproc{ref-mollaioli2014}{{[}9{]}} studied the
influence of \(T_p\) on the DMF for two sets of records: pulse-like and
ordinary records. They found that the DMF for pulse-like records usually
have a pronounced peak or valley located at a period value about one
second less than the pulse and that the ordinates are slightly higher
for ordinary records than for pulse-like records. Neglecting the pulse
period can lead to a significant overestimation for high \(T/T_p\)
values, and they also found that for ordinary records the influence of
fault distance is low to negligible compared to magnitude. Finally, they
proposed an estimator based on the functional form used by
Stafford\citeproc{ref-stafford2008}{{[}10{]}} and
Hatzigeorgiou\citeproc{ref-hatzigeorgiou2010}{{[}11{]}}:

\[DMF = 1 - (5 - \xi)[(1 - \mathbb{I}_{\xi<5\%})(1 - c_6 T^{-0.15})] \times (1 + c_1 \ln \xi + c_2 \ln^2 \xi)(c_3 + c_4 \ln T + c_5 \ln^2 T)\]

where damping is expressed in percentages and the indicator variable
\(\mathbb{I}_{\xi<5\%}\) = 1 if \(\xi < 5\%\) else 0, and the
coefficients depend

on the value of \(T_p\). This estimator is suitable for soil types B, C,
and D, and magnitudes 5 to 7.6 and \(T_p\) from 0.4 to 9s.

Rezaeian et al.~(2012) developed a DMF estimator for the median and
logarithmic standard deviations of shallow crustal earthquakes with a
similar functional form as \citeproc{ref-arias1962}{{[}4{]}}. They
examined the influence of the following predictor variables on the DMF:
the damping ratio, spectral period, ground motion duration, moment
magnitude, source-to-site distance, and site conditions.

They found duration to have the strongest effect on the DMF (besides
damping and period), which is captured in their model by the magnitude
and distance, yielding a total of 4 parameters. Furthermore, they showed
the strength of their predictor against others in the literature.

They mention that near fault effects such as directivity can have a
significant influence on the DMF and that these effects are for the
moment only included implicitly.

Using observations about the physical character and peculiarities of
near-fault pulse-like (NFPL) and surveying different ground motion
models in the literature,

Mavroeidis and Papageorgiou\citeproc{ref-mavroeidis2004}{{[}12{]}}
introduced the so-called MP-wavelets. These have a versatile form
controlled by a few parameters directly related to the physical
phenomena of the rupture, and, unlike some of the simplified pulse
models\citeproc{ref-babak2001}{{[}13{]}},
\citeproc{ref-yang2010}{{[}14{]}}, or the cycloidal-type pulse
models\citeproc{ref-makris2000}{{[}15{]}} that were previously
developed, their\citeproc{ref-mavroeidis2003}{{[}16{]}} parametric
formulation proposed is simpler and physically sound. These MP-wavelets
are highly versatile and can be used to accurately simulate the
classical wavelets such as Ricker, Morlet, or the main velocity pulse of
realistic ground motions (Figure 1) by tuning a reduced number of
parameters with clear physical meaning.

\begin{figure}
\centering
\pandocbounded{\includegraphics[keepaspectratio,alt={Example of the versatility of the MP wavelet in simulating the classical wavelets such as the Gabor, Morlet, DaubechiesDb8, Ricker, Mexican hat, DaubechiesDb8, and others.}]{./md/media/wavelet-zoo.pdf}}
\caption{Example of the versatility of the MP wavelet in simulating the
classical wavelets such as the Gabor, Morlet, DaubechiesDb8, Ricker,
Mexican hat, DaubechiesDb8, and others.}
\end{figure}

Another fundamental difference between a classical wavelet and the
MP-wavelet is that the response of an elastic viscously damped
single-degree-of-freedom system to such pulse can be obtained
\emph{analytically}\citeproc{ref-alonso2015}{{[}17{]}}, which allows for
simpler and more insightful parametric studies or sensitivity analyses.
This contrasts with the classical wavelet treatment whose solution can
only be obtained numerically.

Since real ground motions close to fault with forward directivity
effects have a dominant velocity pulse, it should be possible to
accurately predict the DMF for such motions using the MP-wavelet model;
however, relatively few studies on this matter have been performed to
date. This is the main motivation for the development of this model.

On this matter, Gordó and Miranda \citeproc{ref-gordo2018}{{[}18{]}}
looked at the effect of pulse duration on the DMF and used the MP model
to represent the main pulse of the strong motion waveform of events in
Spain and to predict the response of both single and
multiple-degree-of-freedom systems with high accuracy
\citeproc{ref-alonso2015}{{[}17{]}}. This shows that MP wavelets could
in principle be used to

Given the previous studies, we would like to present our contribution,
which has the following goals:

To develop a simpler model consistent with structural dynamics and
empirical observations, based solely on the spectral shape to estimate
the DMF for MP wavelets.

To prove the effectiveness of the predictor, exploring to what extent
the information on the DMF is present in the spectral ratio.

\section{Definition of Simplified Near-Fault Pulse
Models}\label{definition-of-simplified-near-fault-pulse-models}

\subsection{Definition of an MP wavelet and closed-form damped
displacement
response}\label{definition-of-an-mp-wavelet-and-closed-form-damped-displacement-response}

The velocity waveform of an MP pulse starting at time \(t = 0\)s
consists of a harmonic oscillation component multiplied by a bell-shaped
function:

\[v_p(t) = \frac{A}{2}\cos(\omega t - \pi\gamma + \nu)\left[ 1 - \cos\frac{\omega t}{\gamma} \right]\qquad 0 \leq t \leq t_p = 2\pi\gamma/\omega.\]

In this formulation, \(A\) is the pulse amplitude, which varies from 70
to 130 cm/s (Mavroeidis and Papageorgiou, 2003). The circular frequency
\(\omega = 2\pi f_p\) depends on the moment magnitude of the earthquake
and correlates with the faulting process parameter known as `rise time'
(Mavroeidis and Papageorgiou, 2003). Parameter \(\gamma > 1\) controls
the pulse modulation and fixes the pulse duration
\(t_p = \gamma / f_p = 2\pi\gamma / \omega\). The phase angle \(\nu\)
relates to fault rupture processes and serves as a free parameter for
fitting. Mavroeidis and Papageorgiou (2004) demonstrated that these
pulses accurately simulate NFPL ground motion velocity waveforms.
Several strategies exist for fitting the pulse to real strong motion
records (Mavroeidis and Papageorgiou, 2003; Gordó and Miranda, 2018).
The analytical model can adequately reproduce recorded velocity,
acceleration, or displacement time histories. Alternatively, parameters
can optimize response spectrum reproduction. A simultaneous time-history
and spectra fit (Mavroeidis and Papageorgiou, 2003) often improves
fitting accuracy.

This model focuses on intermediate-to-long period features of NFPL
ground motions. Fitted pulses cannot accurately reproduce responses for
ground motions with stochastic high-frequency content (Mavroeidis and
Papageorgiou, 2003). When the normalized period becomes small
(\(T/T_p \leq 0.83\)), system response primarily depends on
high-frequency components that are difficult to capture (Hubbard and
Mavroeidis, 2011).

Differentiating the velocity equation with respect to time yields the
acceleration time history. Alonso and Miranda (2015) found an equivalent
expression that avoids complex sinusoidal multiplication. The velocity
waveform becomes a linear combination of three cosine pulses with
distinct amplitudes, frequencies, and phases. This equivalency produces
the acceleration expression:

\[a_p(t) = \frac{A\omega}{4}\left\lbrack \frac{\gamma + 1}{\gamma}\sin\left( \frac{\gamma + 1}{\gamma}\omega t - \pi\gamma + \nu \right) + \frac{\gamma - 1}{\gamma}\sin\left( \frac{\gamma - 1}{\gamma}\omega t - \pi\gamma + \nu \right) - 2\sin(\omega t - \pi\gamma + \nu) \right\rbrack\]

Using the velocity equation, Alonso and Miranda (2015) deduced the
closed-form solution for a damped oscillator with frequency \(\Omega\)
and critical damping level \(\xi\). The exact displacement response is:

\[
u(t) = \begin{cases}
u_1(t) = e^{- \xi\Omega t}(C_1\sin\tilde{\Omega}t + C_2\cos\tilde{\Omega}t) + \sum_{i = 1}^{3}C_{3i}\sin(\omega_i t + \phi_i) + C_{4i}\cos(\omega_i t + \phi_i) & \text{if} \ \ \ 0 \leq t \leq t_p \\
u_2(t) = e^{- \xi\Omega(t - t_p)}\left( \tilde{\Omega}^{- 1}\left( \dot{u}_1(t_p) + \xi\Omega u_1(t_p) \right)\sin\tilde{\Omega}(t - t_p) + u_1(t_p)\cos\tilde{\Omega}(t - t_p) \right) & \text{if} \ \ \ t > t_p
\end{cases}
\]

where \(\tilde{\Omega} = \Omega\sqrt{1 - \xi^2}\), the pulse duration
equals \(t_p = 2\pi\gamma / \omega\), and the constants \(C_{3i}\),
\(C_{4i}\) depend on the amplitude, frequency, and phase of each of the
three pulses:

\[C_{3i} = - \frac{(\Omega^2 - \omega_i^2)A_i}{\Omega^4 + \omega_i^4 + 2\omega_i^2\Omega^2(2\xi^2 - 1)}\]

\[C_{4i} = \frac{2\xi\Omega\omega_iA_i}{\Omega^4 + \omega_i^4 + 2\omega_i^2\Omega^2(2\xi^2 - 1)}\]

Finally, constants \(C_2\) and \(C_1\) become:

\[C_2 = \sum_{i = 1}^{3}\frac{(\Omega^4 - \omega_i^4)\sin\phi_i - 2\xi\omega_i\Omega \cos\phi_i}{\Omega^4 + \omega_i^4 + 2\omega_i^2\Omega^2(2\xi^2 - 1)}A_i\]

\[C_1 = \frac{1}{\tilde{\Omega}}\left( \xi\Omega C_2 + \sum_{i = 1}^{3}\left( - \omega_iC_{3i}\sin\phi_i + \omega_iC_{4i}\cos\phi_i \right) \right)\]

\section{Computation of the Damping Modification
Factor}\label{computation-of-the-damping-modification-factor}

One of the original motivations for developing analytic representations
of near-fault pulses was to enable parametric studies or sensitivity
analyses. For the purposes of this work, we compute the damping
modification factor for multiple single-degree-of-freedom oscillators
and pulses. This requires obtaining the maximum absolute displacement
over the complete time-history response, plus a sufficiently long free
vibration phase. This value can occur during any peak in either the
forced or the free phases. Local and global maxima can be found by
setting the derivative of the displacement equation to zero; however the
only way to compute the global absolute maximum for non-convex functions
is by evaluating those critical points and comparing them. This is
computationally equivalent to evaluating \(u(t)\) at sufficiently
closely spaced intervals \(t_i\) and then taking the maximum absolute
value.

This work computed the damping modification factor using the latter
strategy with \(T \in [0, 10]\) s divided in steps \(dT = 0.01\) s:

\[DMF(T, \xi) = \frac{\max_t \mid u(t,  T,  \xi) \mid}{\max_t \mid u(t,  T,  \xi = 5\%) \mid}\]

Structural dynamics dictates that the DMF approaches 1 as the system
becomes infinitely stiff or infinitely flexible, due to damping effects
vanishing in the limit. The displacement amplitude becomes immaterial to
the DMF because it appears in both numerator and denominator. The
displacement equation shows that the response amplitude depends linearly
on \(A\); therefore, this parameter does not affect the DMF and is set
to 1 for this study.

The fundamental frequency of the MP pulse \(f_p\) scales the time axis
and amplitude proportionally to \(1/f_p\), skewing the spectra's
abscissas and ordinates while preserving the underlying functional form.
This scaling property implies that one can reconstruct an equivalent
spectrum (and consequently an equivalent DMF) for a different pulse
frequency by scaling a unitary frequency spectrum (\(f_p = 1 = T_p\))
appropriately (see Appendix B for a detailed explanation).

This study sets \(f_p = 1\) for simplicity. In general, the pulse period
differs from 1, and ground motion normalization should use \(T/T_p\).
This study takes the pulse period \(T_p\) as the period at the peak of
the response pseudo-velocity spectrum for \(\xi = 0.05\), although more
sophisticated methods based on energy or wavelet theory exist (Baker,
2007; Mollaioli and Bosi, 2012). Consequently, without loss of
generality, the only explicit parameters for MP-wavelets in this work
are the modulating frequency \(\gamma\) and the phase shift \(\nu\).

These considerations led to a parametric study of displacement damping
modification factors for a set of 66 representative pulses (see Appendix
A). Figure 2 presents a subset of these results.

As previously observed (Hubbard and Mavroeidis, 2011), when \(\gamma\)
approaches 1, DMFs plateau around \(f_p\). As \(\gamma\) increases, DMFs
spike near \(T/T_p = 1\) due to resonance. Values of \(\gamma\) above 3
contain too many cycles to represent real strong near-fault ground
motion records and are excluded from this study (Mavroeidis and
Papageorgiou, 2004). The response in the very short period range becomes
highly erratic.

For all wavelets, especially with low \(\gamma\)s, there are some
normalized periods where the DMF is not smooth, that is, the derivative
is discontinuous (see, for instance, first row and second column on
Figure 2 at around \(T/T_p\) equal to 1 and 2.3), which indicates a
qualitative change in the behavior of the response. Note that this is
also the case for the very short period showing signs of `ringing'
alternating around the value 1.

This phenomenon has been studied in shock spectra for simple
pulses\citeproc{ref-chopra2012}{{[}19{]}}. The explanation is that for
the period of interest, multiple identical peaks develop both in the
forced and the free vibration phase of the response. Those peaks do not
behave equally when the period of the system is perturbed slightly,
leading to the observed loss of smoothness.

This phenomenon is not easy to capture in a DMF model.

\begin{figure}
\centering
\pandocbounded{\includegraphics[keepaspectratio,alt={Table of DMFS for some selected MP pulses (parameters shown above each sub-figure), for damping levels of \textbackslash xi = 0.01, 2, 4, 6, 8, 10, 20\%. All values were computed with f\_p = T\_p = 1.}]{./md/media/dmf_by_damping.pdf}}
\caption{Table of DMFS for some selected MP pulses (parameters shown
above each sub-figure), for damping levels of
\(\xi = 0.01, 2, 4, 6, 8, 10, 20%
\). All values were computed with \(f_p = T_p = 1\).}
\end{figure}

\section{Spectral Shape Measures as Predictors for Damping Modification
Factors}\label{spectral-shape-measures-as-predictors-for-damping-modification-factors}

It is conjectured that a given response spectrum contains sufficient
information about the dynamical response of a system to accurately
predict changes in its response due to changes in damping levels, thus
making it a potentially useful and practical predictor of DMF. The
complex dependencies of seismological parameters with respect to the
response of SDOF systems, such as magnitude, fault distance, directivity
are thought to be contained within the shape and ordinate values of the
spectrum; thus, we seek measures to exploit this information for
prediction purposes. One of the first measures of spectral shapes
proposed is the so-called `SaRatio' \citeproc{ref-miranda2020}{{[}6{]}}
which is based solely on the 5\% critically-damped response spectra
discrete data points. The SaRatio is defined as the 5\% critically
damped spectral ordinate at the oscillator period \(T\), normalized
pointwise by the geometric mean of the 5\% damping ratio spectral
ordinates over the window \([aT, bT]\):

\[S_R(T,a,b) = \frac{S(T,\xi = 5\%)}{\bar{S}([aT,bT])}\]

The denominator represents a ``running geometric average'' with variable
window:

\[\bar{S}([aT,bT]) = \left( \prod_{i = 0}^{N - 1} S\left( (a + \frac{i}{N}(b - a))T,\xi = 5\% \right) \right)^{1/N}\]

where \(N\) equals the number of equally spaced ordinates. To evaluate
\(\bar{S}\) at period \(T\), we compute the geometric average in
interval \([aT, bT]\) around \(T\) for all \(N\) equally spaced spectrum
ordinates. The limiting values become 1 as \(a\) and \(b\) approach 1,
and the ratio of original spectral ordinates to the total geometric mean
as \(a = 0\) and \(b \to \infty\). This work uses spectral displacement
for computing \(S_R\) due to numerical stability.

Alternative shape measures could use harmonic or weighted means, or
signal processing techniques to exploit spectrum shape information.
These possibilities remain unexplored, and the geometric mean choice
lacks complete understanding.

Miranda and Miranda (2020) found that for pseudo-acceleration spectra
predicting DMF for Chilean interface records, the optimum \(a\) value
ranges between 0.02 and 0.98 with high variability, while \(b\) shows
lower variability with optimum values between 1.1 and 1.4.

\subsection{Period-implicit scatter plots for all representative
wavelets}\label{period-implicit-scatter-plots-for-all-representative-wavelets}

This study investigates the correlation between SdRatio \(S_R\) and DMF
for a full suite of representative MP pulses (see Appendix A), seeking
\(a, b\) values that yield the best predictive model for DMF at given
period and damping.

We computed damping modification factors for 14 damping levels:
\(\xi = 0.01 \approx 0, 1, 2, 3, 4, 6, 7, 8, 9, 10, 15, 20, 30\%\). The
period range spans 0.01 to 10s in 0.01s intervals to compute \(S_R\) for
high \(b\) values. We chose spectral displacement as the spectral ratio
function \(S_R\).

We selected 30 parameter combinations from
\(a \in [0, 0.02, 0.25, 0.5, 0.75, 0.98] \times b \in [1.02, 1.2, 1.5, 2.0, 2.5]\).
This represents a compromise between optimal limits proposed by Miranda
and Miranda (2020) and a sufficiently small, well-spaced parameter space
for exploration.

We plotted \((S_R, \text{DMF})\) pairs against each other for all
wavelets and periods across selected \((\xi, a, b)\) combinations.
Period dependency becomes implicit as all spectral ordinates appear as
points.

Figure 3 reveals that some scatter plots exhibit linear or quadratic
relationships.

\begin{figure}
\centering
\pandocbounded{\includegraphics[keepaspectratio,alt={Comparison of different period-implicit scatter plots for different selections of a, b and \textbackslash xi (in percentage), resulting in a wide variety of scatter forms. Note the seemingly linear and quadratic relationships, alongside interesting branching behavior, e.g., second row second column trace.}]{./md/media/grid-scatters-paper.pdf}}
\caption{Comparison of different period-implicit scatter plots for
different selections of \(a, b\) and \(\xi\) (in percentage), resulting
in a wide variety of scatter forms. Note the seemingly linear and
quadratic relationships, alongside interesting branching behavior, e.g.,
second row second column trace.}
\end{figure}

Analysis of Figure 3 shows that some \((a, b)\) combinations---top-right
or bottom-left---produce traces following linear or quadratic
relationships. Values of \((a, b)\) close to 1 cause \(S_R\) to branch
(see second-row second-column trace or third-row first-column trace),
which are not useful because the window is too narrow.

The density of points near (1, 1) becomes extreme due to limiting values
at both DMF ends for all wavelets. This suggests employing a weighted or
robust regression predictor.

\section{Proposed DMF Predictor}\label{proposed-dmf-predictor}

Based on these observations, we consider a simple model to predict the
DMF for \emph{any} MP wavelet:

\[DMF(T, \xi) = \theta(\xi) (S_R(T) - 1) + 1\]

This model captures fundamental physical facts about the DMF and its
shape. It has correct limiting values: as \(S_R \to 1\), \(DMF \to 1\).

We performed weighted least squares regression based on observations
from the previous section and Figure 2. We compared model accuracy
across sample waveforms and damping levels using root-mean-squared-error
(RMSE) as the measure. RMSE quantifies the dispersion of the estimator
\(\widehat{y}\) around the true value \(y_i\):

\[RMSE = \sqrt{\mathbb{E}[(y - \widehat{y})^{2}]} \approx \sqrt{\frac{1}{N}\sum_{i = 1}^{N}(y_i - \widehat{y_i})^{2}}\]

The median RMSE across waveforms measures the median dispersion or
variance for fixed \((a, b)\) levels.

We seek \((a, b)\) values that yield the lowest median RMSE with low
variance. These values are independent of \(T_p\), although their
physical significance remains unclear.

Figure 4 shows the distribution of RMSEs across all wavelets and damping
values as a boxplot for the 8 \((a, b)\) combinations with lowest median
errors.

The middle line inside the box indicates the median. The box height
represents the interquartile range---the difference between upper
quartile (0.75) and lower quartile (0.25). The whiskers extend 1.5 times
the interquartile range from the upper and lower quartiles. Outliers
appear as small points and relate to damping levels close to 0.

\begin{figure}
\centering
\pandocbounded{\includegraphics[keepaspectratio,alt={Boxplot for the RMSEs achieved by our model for the 8 selections of (a, b) which yielded the lowest median error across all damping values. Most combinations have low median RMSEs.}]{./md/media/box-plot-paper.pdf}}
\caption{Boxplot for the RMSEs achieved by our model for the 8
selections of \((a, b)\) which yielded the lowest median error across
all damping values. Most combinations have low median RMSEs.}
\end{figure}

The \((a, b)\) values yielding lowest median RMSE are \(a = 0\) and
\(b = 2.0\). These values are also the simplest, which might indicate
underlying physical significance.

Table 1 gives the regression values for the slope parameter
\(\theta(\xi)\) of the chosen model.

\begin{longtable}[]{@{}
  >{\raggedright\arraybackslash}p{(\linewidth - 20\tabcolsep) * \real{0.1379}}
  >{\raggedright\arraybackslash}p{(\linewidth - 20\tabcolsep) * \real{0.0862}}
  >{\raggedright\arraybackslash}p{(\linewidth - 20\tabcolsep) * \real{0.0862}}
  >{\raggedright\arraybackslash}p{(\linewidth - 20\tabcolsep) * \real{0.0862}}
  >{\raggedright\arraybackslash}p{(\linewidth - 20\tabcolsep) * \real{0.0862}}
  >{\raggedright\arraybackslash}p{(\linewidth - 20\tabcolsep) * \real{0.0862}}
  >{\raggedright\arraybackslash}p{(\linewidth - 20\tabcolsep) * \real{0.0862}}
  >{\raggedright\arraybackslash}p{(\linewidth - 20\tabcolsep) * \real{0.0862}}
  >{\raggedright\arraybackslash}p{(\linewidth - 20\tabcolsep) * \real{0.0862}}
  >{\raggedright\arraybackslash}p{(\linewidth - 20\tabcolsep) * \real{0.0862}}
  >{\raggedright\arraybackslash}p{(\linewidth - 20\tabcolsep) * \real{0.0862}}@{}}
\caption{Slope values for the proposed model as a function of damping
value. The maximum damping value recommended for use is 0.30. Use linear
interpolation for values not listed explicitly.}\tabularnewline
\toprule\noalign{}
\begin{minipage}[b]{\linewidth}\raggedright
\(\xi\)
\end{minipage} & \begin{minipage}[b]{\linewidth}\raggedright
0.01
\end{minipage} & \begin{minipage}[b]{\linewidth}\raggedright
0.02
\end{minipage} & \begin{minipage}[b]{\linewidth}\raggedright
0.03
\end{minipage} & \begin{minipage}[b]{\linewidth}\raggedright
0.04
\end{minipage} & \begin{minipage}[b]{\linewidth}\raggedright
0.06
\end{minipage} & \begin{minipage}[b]{\linewidth}\raggedright
0.08
\end{minipage} & \begin{minipage}[b]{\linewidth}\raggedright
0.10
\end{minipage} & \begin{minipage}[b]{\linewidth}\raggedright
0.15
\end{minipage} & \begin{minipage}[b]{\linewidth}\raggedright
0.20
\end{minipage} & \begin{minipage}[b]{\linewidth}\raggedright
0.30
\end{minipage} \\
\midrule\noalign{}
\endfirsthead
\toprule\noalign{}
\begin{minipage}[b]{\linewidth}\raggedright
\(\xi\)
\end{minipage} & \begin{minipage}[b]{\linewidth}\raggedright
0.01
\end{minipage} & \begin{minipage}[b]{\linewidth}\raggedright
0.02
\end{minipage} & \begin{minipage}[b]{\linewidth}\raggedright
0.03
\end{minipage} & \begin{minipage}[b]{\linewidth}\raggedright
0.04
\end{minipage} & \begin{minipage}[b]{\linewidth}\raggedright
0.06
\end{minipage} & \begin{minipage}[b]{\linewidth}\raggedright
0.08
\end{minipage} & \begin{minipage}[b]{\linewidth}\raggedright
0.10
\end{minipage} & \begin{minipage}[b]{\linewidth}\raggedright
0.15
\end{minipage} & \begin{minipage}[b]{\linewidth}\raggedright
0.20
\end{minipage} & \begin{minipage}[b]{\linewidth}\raggedright
0.30
\end{minipage} \\
\midrule\noalign{}
\endhead
\bottomrule\noalign{}
\endlastfoot
\(\theta\) & 0.093 & 0.062 & 0.036 & 0.016 & 0.015 & 0.045 & 0.076 &
0.156 & 0.238 & 0.404 \\
\end{longtable}

\textbf{Algorithm 1} DMF estimation using spectral ratio

\textbf{Input:} 5\% damped displacement spectrum, target damping ratio
\(\xi\), structural period \(T\)

\textbf{Output:} DMF value for the target damping ratio and period

\begin{enumerate}
\def\labelenumi{\arabic{enumi}.}
\item
  \textbf{Compute spectral ratio:} Calculate \(S_R(T)\) using the
  spectral ratio equation
\item
  \textbf{Interpolate slope parameter:} For target damping ratio
  \(\xi\), interpolate slope parameter \(\theta(\xi)\) from Table 1. Use
  linear interpolation for damping values not explicitly listed
\item
  \textbf{Calculate DMF:} Apply the DMF predictor equation:
  \[DMF(T, \xi) = 1 + \theta(\xi) \cdot (S_R(T) - 1)\]
\item
  \textbf{Return:} The computed DMF value
\end{enumerate}

\section{Goodness-of-fit against
MP-wavelets}\label{goodness-of-fit-against-mp-wavelets}

Figure 5 shows linear fits to four different damping levels, colored by
period.

\begin{figure}
\centering
\pandocbounded{\includegraphics[keepaspectratio,alt={Linear weighted fit for 4 different damping levels colored by period}]{./md/media/scatter_grid.pdf}}
\caption{Linear weighted fit for 4 different damping levels colored by
period}
\end{figure}

Note that this particular shown fit was not forced to pass through (1,1)
for practical purposes it naturally did for almost all dampings, which
shows the adequacy of the model to the data.

\subsection{Direct comparison}\label{direct-comparison}

Figures 6 and 7 compare the exact DMF (black) with predictions from the
proposed model for sample wavelets.

\pandocbounded{\includegraphics[keepaspectratio,alt={Direct comparison of the model against the DMF of sample wavelets for 0\% critical damping.}]{./md/media/direct-0.pdf}}
\pandocbounded{\includegraphics[keepaspectratio,alt={Direct comparison of the model against the DMF of sample wavelets for 10\% critical damping.}]{./md/media/direct-10.pdf}}

The model closely matches the original DMF shapes, even for 0\% critical
damping. The main peak is captured accurately.

The loss of smoothness observed DMF discussed in section 3 partially
captured in \(T/T_p > 0.5\) but is not that accurate in the
high-frequency regime \(T/T_p \in (0,0.2)\). Errors in that `ringing'
area are not too concerning since it is a small regime.

We note that it is somewhat surprising to have such a good fit for a
model with only a single parameter (\(\theta\)) and no underlying pulse
or seismological parameters.

\subsubsection{Analysis of residuals}\label{analysis-of-residuals}

Figure 8 shows residuals between predicted and true DMF as functions of
SdRatio for \(\xi \approx 0\%\). Many points cluster near 1, which
contributes to non-constant variance in the residuals.

A wavy pattern in the plot relates to residuals at the peaks of
different wavelets.

\begin{figure}
\centering
\pandocbounded{\includegraphics[keepaspectratio,alt={Residuals between our predicted DMF and the true DMF as functions of SdRatio for \textbackslash xi \textbackslash approx 0}]{./md/media/residuals_plot.pdf}}
\caption{Residuals between our predicted DMF and the true DMF as
functions of SdRatio for \(\xi \approx 0\)}
\end{figure}

Statistical tests on error homoscedasticity reject the null hypothesis
with \(p\)-values near zero. Power transforms such as Box-Cox or
Box-Tidwell could address heteroscedasticity, but would complicate the
model unnecessarily. Removing data for \(T/T_p < 0.83\) improves
\(p\)-values, but the null hypothesis of homoscedasticity is still
rejected.

\subsubsection{Justification of the
goodness-of-fit}\label{justification-of-the-goodness-of-fit}

This section justifies the model's adequacy despite heteroscedasticity.

The model is used for prediction, not inference, so non-constant
variance is less critical in this case. Mankiw (2016) notes that
heteroscedasticity ``has never been a reason to throw out an otherwise
good model.'' The heteroscedasticity may reveal information about
SdRatio variation at wavelet peaks.

Figures 6 and 7 provide the strongest evidence of goodness-of-fit,
showing that the model faithfully captures DMF shape across all wavelets
and damping levels.

The model's limiting values are physically sound, and the fit passes
naturally through (1,1) without being forced, which supports the model's
validity.

Perhaps unsurprisingly, the proposed model is a good predictor for
MP-wavelets, since it was developed specifically for these pulses. But a
reasonable question arises: how would the presented model perform for
other pulse-like ground motions, none of which were used in its
development? This will be discussed in the following section.

\section{Comparison against other narrow-band ground
motions}\label{comparison-against-other-narrow-band-ground-motions}

This exploration tests our conjecture that most information required to
predict the DMF resides within \(S_R\). If true, the model developed
using only MP-wavelets should perform well for other ground motions,
even with different frequency content.

We employed Tarquis pulses for this comparison. These simple waveforms
represent sinusoidal pulses corrected to become realistic ground motion
representations\citeproc{ref-tarquis1988}{{[}20{]}},
\citeproc{ref-arroyo2007}{{[}21{]}}. Figure 9 shows the three different
Tarquis pulses used in this study. The 20 and 50 second durations exceed
realistic strong ground motion durations but were selected to strain the
proposed model's predictive limits.

\begin{figure}
\centering
\pandocbounded{\includegraphics[keepaspectratio,alt={Tarquis accelerograms for 5, 20, and 50 seconds with unit amplitude and frequency.}]{./md/media/tarquis-pulses.png}}
\caption{Tarquis accelerograms for 5, 20, and 50 seconds with unit
amplitude and frequency.}
\end{figure}

Figure 10 presents four sample DMFs for different damping and pulse
combinations. The proposed model captures the main resonant peak
accurately, except for very low damping levels combined with high
durations, where the peak becomes very sharp. This suggests that
explicit duration inclusion could improve the model.

The erratic non-smooth regime of \(T/T_p < 0.83\) also remains
incompletely captured.

\begin{figure}
\centering
\pandocbounded{\includegraphics[keepaspectratio,alt={Comparison of the true DMF for Tarquis pulses vs the proposed predictor for different critical damping levels.}]{./md/media/tarquis_2x2.pdf}}
\caption{Comparison of the true DMF for Tarquis pulses vs the proposed
predictor for different critical damping levels.}
\end{figure}

\section{Summary and Conclusions}\label{summary-and-conclusions}

The closed-form displacement response of a viscously-damped oscillator
to an MP-pulse was presented; this representation allows for analytic
and parametric studies of the response of SDOF systems to pulse-like
ground motions. It was noted that only the ``modulation'' \(\gamma\) and
the phase shift \(\nu\) of the pulse are required to capture the
complete response variability of a SDOF system to such wavelets. A DMF
estimator based on the spectral shape measure SaRatio \(S_R\) was
proposed that has a sound physical basis, possesses the correct limiting
values for short and long period ranges, and is simpler than others
found in the literature.

We proposed a DMF estimator based on the spectral shape measure SaRatio
\(S_R\) that possesses sound physical basis, correct limiting values for
short and long period ranges, and an elegant simplicity not found in
other literature models.

The proposed estimator requires a single parameter dependent on the
target damping level and uses only the 5\% critically damped
displacement spectrum as input. No record-specific or seismological
parameters are needed.

The estimator produced low median errors and low variance for all
damping ratios across MP-wavelets, including damping levels close to 0.

A predictive study using realistic narrow-band ground motions, the
Tarquis pulses, demonstrated the model's simplicity and power. The model
showed good predictive capability despite not being specifically
developed for these narrow-band records.

This work establishes that spectral shape measures can construct simple
and accurate DMF predictors for real NFPL ground motions, representing
the next logical research step.

The authors are unaware of the physical explanation as to why the
SdRatio is such a good predictor despite being so simple, which implies
that further research is required to explore the reasons and limits of
its predictive power.

\section{Acknowledgments}\label{acknowledgments}

\section{Author contributions}\label{author-contributions}

Carlo Ruiz reviewed and compiled the relevant literature, performed the
computations, generated the figures, and typeset the paper. Mario Ordaz
provided the main research direction and guidance, proposed the
functional forms of the period-implicit predictor (equations for
spectral ratio and geometric mean), and suggested the exploration of its
predictive power against narrow-band motions.

\section{Financial disclosure}\label{financial-disclosure}

The first author would like to acknowledge and express his gratitude to
the Instituto de Ingeniería at UNAM, Mexico, for the financial support
given to him during his doctoral studies. He would also like to
acknowledge and express his gratitude to the SNI, CONAHCYT, Mexico, for
the financial support received during the preparation of the paper.

\section{Conflict of interest}\label{conflict-of-interest}

The authors declare no potential conflict of interest.

\section{Appendix A}\label{appendix-a}

\subsection{Suite of wavelets used in this
study}\label{suite-of-wavelets-used-in-this-study}

We present the set of representative wavelets used in this study (Figure
11),

As explained, \(A\) and \(f_p\) are immaterial to the DMF and a such
only \(\gamma, \nu\) need to be varied. Therefore, for simplicty a total
of \(11\times 6=66\) pulses were constructed from the cartesian product
of

\(\gamma \in [ 1, 1.2, 1.4, 1.6, 1.8, 2, 2.2, 2.4, 2.6, 2.8, 3]\) and
\(\nu \in [0, 1/3 \pi, 2/3 \pi, \pi, 4/3 \pi, 5/3 \pi ]\)

This way, the suite is small enough to be easily reproducible and
includes much of the information content from the wavelets. For any
given NFPL record, we could potentially conceive that one (or
simultaneously many) of such wavelets could approximate its velocity
waveform if fit appropriately.

We excluded pulses with \(\gamma > 3\) as they are not representative of
real near-fault pulse-like waveforms.

\begin{figure}
\centering
\pandocbounded{\includegraphics[keepaspectratio,alt={Sample of the family of 66 representative wavelets used in this study. Note that higher \textbackslash gamma lead to more modulation and different values of \textbackslash nu affect the phase of the pulses.}]{./md/media/pulse-suite.pdf}}
\caption{Sample of the family of 66 representative wavelets used in this
study. Note that higher \(\gamma\) lead to more modulation and different
values of \(\nu\) affect the phase of the pulses.}
\end{figure}

\section{Appendix B}\label{appendix-b}

\subsection{\texorpdfstring{Proof of \(f_p\) being immaterial to the DMF
of
MP-wavelets}{Proof of f\_p being immaterial to the DMF of MP-wavelets}}\label{proof-of-f_p-being-immaterial-to-the-dmf-of-mp-wavelets}

both an analytical and a geometric/visual proof.

\protect\phantomsection\label{refs}
\begin{CSLReferences}{0}{0}
\bibitem[\citeproctext]{ref-kawasumi1956}
\CSLLeftMargin{{[}1{]} }%
\CSLRightInline{H. Kawasumi and K. Kanai, {``Vibrations of buildings in
japan (in two parts), part i: Small amplitude vibrations of actual
buildings,''} \emph{Earthquake Research Institute, Tokyo Japan}, 1956.}

\bibitem[\citeproctext]{ref-hudson1956}
\CSLLeftMargin{{[}2{]} }%
\CSLRightInline{D. Hudson, {``Response spectrum techniques in
engineering seismology,''} in \emph{World conference on earthquake
engineering, berkeley california}, 1956.}

\bibitem[\citeproctext]{ref-newmark1982}
\CSLLeftMargin{{[}3{]} }%
\CSLRightInline{N. Newmark and W. Hall, \emph{Earthquake spectra and
design}. Earthquake Engineering Research Institute Berkeley, CA, 1982.}

\bibitem[\citeproctext]{ref-arias1962}
\CSLLeftMargin{{[}4{]} }%
\CSLRightInline{A. Arias and R. Husid, {``Influencia del amortiguamiento
sobre la respuesta de estructuras sometidas a temblor,''} \emph{Revista
IDIEM}, vol. 1, no. 3, pp. 219--228, 1962.}

\bibitem[\citeproctext]{ref-rosenblueth1964}
\CSLLeftMargin{{[}5{]} }%
\CSLRightInline{E. Rosenblueth, A. Arias, and R. Husid, {``Discusi{ó}n
del art{í}culo de a. Arias y r. Husid titulado {`influencia de
amortiguamiento sobre la respuesta estructuras sometidas a temblor'},''}
\emph{Revista IDIEM}, vol. 3, no. 1, pp. 63--69, 1964.}

\bibitem[\citeproctext]{ref-miranda2020}
\CSLLeftMargin{{[}6{]} }%
\CSLRightInline{S. Miranda, E. Miranda, and J. Llera, {``The effect of
spectral shape on damping modification factors,''} \emph{Earthquake
Spectra}, vol. 36, no. 4, pp. 2086--2111, 2020.}

\bibitem[\citeproctext]{ref-hubbard2011}
\CSLLeftMargin{{[}7{]} }%
\CSLRightInline{D. Hubbard and G. Mavroeidis, {``Damping coefficients
for near-fault ground motion response spectra,''} \emph{Soil Dynamics
and Earthquake Engineering}, vol. 31, no. 3, pp. 401--417, 2011.}

\bibitem[\citeproctext]{ref-pu2016}
\CSLLeftMargin{{[}8{]} }%
\CSLRightInline{W. Pu, K. Kasai, E. Kabando, and B. Huang, {``Evaluation
of the damping modification factor for structures subjected to
near-fault ground motions,''} \emph{Bulletin of Earthquake Engineering},
vol. 14, no. 6, pp. 1519--1544, 2016.}

\bibitem[\citeproctext]{ref-mollaioli2014}
\CSLLeftMargin{{[}9{]} }%
\CSLRightInline{F. Mollaioli, L. Liberatore, and A. Lucchini,
{``Displacement damping modification factors for pulse-like and ordinary
records,''} \emph{Engineering Structures}, vol. 78, pp. 17--27, 2014.}

\bibitem[\citeproctext]{ref-stafford2008}
\CSLLeftMargin{{[}10{]} }%
\CSLRightInline{P. Stafford, R. Mendis, and J. Bommer, {``Dependence of
damping correction factors for response spectra on duration and numbers
of cycles,''} \emph{Journal of structural engineering}, vol. 134, no. 8,
pp. 1364--1373, 2008.}

\bibitem[\citeproctext]{ref-hatzigeorgiou2010}
\CSLLeftMargin{{[}11{]} }%
\CSLRightInline{G. Hatzigeorgiou, {``Damping modification factors for
SDOF systems subjected to near-fault, far-fault and artificial
earthquakes,''} \emph{Earthquake Engineering \& Structural Dynamics},
vol. 39, no. 11, pp. 1239--1258, 2010.}

\bibitem[\citeproctext]{ref-mavroeidis2004}
\CSLLeftMargin{{[}12{]} }%
\CSLRightInline{G. Mavroeidis, G. Dong, and A. Papageorgiou,
{``Near-fault ground motions, and the response of elastic and inelastic
singledegree-of-freedom (SDOF) systems,''} \emph{Earthquake Engineering
\& Structural Dynamics}, vol. 33, no. 9, pp. 1023--1049, 2004.}

\bibitem[\citeproctext]{ref-babak2001}
\CSLLeftMargin{{[}13{]} }%
\CSLRightInline{A. Babak and H. Krawinkler, {``Effects of near fault
ground motions on framed structures,''} The John A. Blume Earthquake
Engineering Research Center, Stanford University, 2001.}

\bibitem[\citeproctext]{ref-yang2010}
\CSLLeftMargin{{[}14{]} }%
\CSLRightInline{D. Yang, J. Pan, and G. Li, {``Interstory drift ratio of
building structures subjected to near-fault ground motions based on
generalized drift spectral analysis,''} \emph{Soil Dynamics and
Earthquake Engineering}, vol. 30, no. 11, pp. 1182--1197, 2010.}

\bibitem[\citeproctext]{ref-makris2000}
\CSLLeftMargin{{[}15{]} }%
\CSLRightInline{N. Makris and S. Chang, {``Response of damped
oscillators to cycloidal pulses,''} \emph{Journal of Engineering
Mechanics}, vol. 126, no. 2, pp. 123--131, 2000.}

\bibitem[\citeproctext]{ref-mavroeidis2003}
\CSLLeftMargin{{[}16{]} }%
\CSLRightInline{G. Mavroeidis and A. Papageorgiou, {``A mathematical
representation of near-fault ground motions,''} \emph{Bulletin of The
Seismological Society of America}, vol. 93, no. 3, pp. 1099--1131,
2003.}

\bibitem[\citeproctext]{ref-alonso2015}
\CSLLeftMargin{{[}17{]} }%
\CSLRightInline{A. Alonso-Rodríguez and E. Miranda, {``Assessment of
building behavior under near-fault pulse-like ground motions through
simplified models,''} \emph{Soil Dynamics and Earthquake Engineering},
vol. 79, pp. 47--58, 2015.}

\bibitem[\citeproctext]{ref-gordo2018}
\CSLLeftMargin{{[}18{]} }%
\CSLRightInline{C. Gordó Monsó and E. Miranda, {``Significance of
directivity effects during the 2011 lorca earthquake in spain,''}
\emph{Bulletin of Earthquake Engineering}, vol. 16, no. 12, pp. 1--20,
2018.}

\bibitem[\citeproctext]{ref-chopra2012}
\CSLLeftMargin{{[}19{]} }%
\CSLRightInline{A. Chopra, \emph{Dynamics of structures}, 4th ed.
Prentice Hall, 2012.}

\bibitem[\citeproctext]{ref-tarquis1988}
\CSLLeftMargin{{[}20{]} }%
\CSLRightInline{F. Tarquis, {``Structural response and design spectra
for the 1985 mexico city earthquake,''} PhD thesis, University of Texas,
Austin, 1988.}

\bibitem[\citeproctext]{ref-arroyo2007}
\CSLLeftMargin{{[}21{]} }%
\CSLRightInline{D. Arroyo and M. Ordaz, {``Use of corrected sinusoidal
pulses to estimate inelastic demands of elasto-perfectly plastic
oscillators subjected to narrow-band motions,''} \emph{Journal of
Earthquake Engineering}, vol. 11, no. 3, pp. 303--325, 2007.}

\end{CSLReferences}
